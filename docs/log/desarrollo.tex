\chapter{Desarrollo}
\minitoc
\label{chap:desarrollo}
Se ha seguido un proceso de desarrollo en espiral con prototipado. El diseño y desarrollo de los diferentes componentes del sistema se presta a este tipo de ciclo ya que permite crear y probar en cada iteración un producto prototipo, centrándose en las primeras iteraciones en diseñar más e implementar menos y en las últimas solo refinar el software final.

No obstante dentro de este ciclo en espiral se han diferenciado varias etapas correspondientes a cada uno de los módulos del mismo.

\section{Iteración 1}
Se empezó por identificar los formatos, herramientas y componentes del proyecto. Como software de entrada para las partituras se escogió MuseScore, principalmente por ser \textit{opensource} y por su sencillez. MuseScore no posee una curva de aprendizaje difícil, ya que la introducción de las notas se puede hacer de forma visual en su interfaz mediante el ratón o con el teclado y no requiere mayor preparación para crear y exportar una pieza musical sencilla. Su soporte para \textit{plugins} también resulta interesante, ya que el proyecto podría llegar a transformarse en un \textit{plugin} para la herramienta si se desease una mayor integración con la misma. MuseScore además ofrece soporte para los tres formatos de ficheros musicales contemplados, así como para PDF y otros formatos finales de imagen.

Para el formato de entrada y salida, se compararon las propiedades de MIDI, LilyPond y MusicXML. Los tres formatos ofrecen posibilidades de edición, aunque cada uno sirve a un propósito diferente. MIDI no trabaja con ficheros textuales, sino que codifica de forma binaria toda la información de los eventos de la canción, LilyPond es un formato de texto plano pensado para que una persona pueda editarlo a mano, posee una estructura de marcado a través de etiquetad de solo apertura o autocerradas y trabaja con identaciones para formar la jerarquía del fichero, esto requiere al usuario escribir mucho menos, pero puede suponer problemas para un \textit{parser} convencional debido a que la identación de LilyPond no está estandarizada. Por último MusicXML se presenta como el formato idóneo para la tarea, ya que al ser una extensión de XML, está orientado a que una máquina pueda leerlo y crear una estructura en memoria con tod ala información que necesita para poder extraer los datos de la partitura. Además, la implementación de un \textit{parser} de un lenguaje etiquetado como XML es un problema convencional y fácilmente abarcable.

En cuanto a la tecnología escogida para diseñar y construir el \textit{parser} se optó por las bibliotecas Flex y Bison para C.

Analizando la especificación del esquema de MusicXML de cara a desarrollar el \textit{parser} se identificaron las diferentes partes del mismo. MusicXML incluye información tanto musical como de representación gráfica de los diferentes elementos de la partitura. Todos esta información gráfica es generada automáticamente por el software que exporta el fichero MusicXML (MuseScore para el caso) y resulta irrelevante a la hora de extraer los hechos lógicos presentes en la partitura, por tanto se decidió obviarla. 

MusicXML declara inicialmente el número de voces presente en la partitura mediante la etiqueta part-list y sus etiquetas anidadas score-part. Más adelante, las etiquetas part se encargan de contener los compases mediante la etiqueta measure que a su vez contienen las etiquetas note. Son estas últimas etiquetas las que hay que desglosar para extraer los hechos lógicos, aunque la información de las etiquetas measure también es relevante, así como poder asignar un identificador a cada voz de la partitura para poder diferenciarlas a nivel lógico.

\begin{verbatim}
<note default-x="74.65" default-y="-25.00">
    <pitch>
        <step>A</step>
        <octave>4</octave>
    </pitch>
    <duration>1</duration>
    <voice>1</voice>
    <type>quarter</type>
    <stem>up</stem>
</note>
\end{verbatim}

La etiqueta note posee dos parámetros visuales, default-x y default-y,  que indican su posición en coordenadas x e y, como ya se ha comentado antes, esta información es meramente visual. La etiqueta pitch describe el sonido de la nota mediante el nombre de la nota usando notación internacional y la octava de la nota. Duration especifica la duración en tiempos del compás de la nota actual, voice asigna este sonido a la voz correspondiente, type indica la figura mediante el sistema fraccionario y stem dice si la plica de la nota es ascendente o descendente, de nuevo esto es meramente visual.

Ya que el armonizador solo contemplará partituras uniformes (es decir, con un solo tipo de figura presente), unicamente necesitamos saber la voz, el compás, el tiempo en el que ocurre la nota relativo al compás y el sonido de la nota.

Con esta información en mente se procedió a desarrollar el \textit{parser} de MusicXML a hechos en ASP. Se hizo uso de Flex para el análisis léxico y de Bison para el análisis sintáctico. El primer paso fue identificar los diferentes tags relevantes para el análisis léxico, estos son note para las notas, step para el ritmo, rest para los silencios, chord para los acordes, octave para la octava de cada nota, alteration para identificar sostenidos y bemoles, part para las diferentes voces de la pieza y por último elementos relevantes para el análisis de XML pero carentes de significado real para los hechos lógicos de la partitura tales como la apertura y cierre de etiquetas, campos textuales o símbolos varios.

En el código de Bison se diseñó la gramática primero y se integró junto con el código C encargado de tomar los datos extraídos por el \textit{parser} y exportarlos a un fichero de hechos lógicos listo para ser interpretado. La gramática diseñada para ello parte de una regla inicial que puede derivar en un bloque compuesto por etiquetas anidadas o en un error si el fichero no tiene el formato y estructura adecuados.

Los bloques de etiquetas se descompusieron en tokens según su contenido o en dos partes (part1 y part2) dependiendo de si la etiqueta es autocerrada o no. Los errores que se pueden encontrar en este punto son que la etiqueta se abra pero no se cierre o que haya elementos de más no reconocidos tras la segunda parte de la misma.

Las dos partes de cada etiqueta no autocerrada se corresponden respectivamente con la apertura de la etiqueta (part1) y con el contenido y cierre de la misma (part2).

Por último se hace uso de una regla recursiva que permite anidar los bloques de etiquetas y contenido.

Las pruebas realizadas al \textit{parser} revelaron que existía un problema de análisis al no poder verificar de forma sencilla que cada etiqueta se cerraba de modo correcto, es decir, que el nombre de la etiqueta que cierra un bloque sea el mismo del que la abrió, se implementó una pila en C para esta tarea. La implementación se realizó de modo que la pila no tuviese un tipo definido de partida, por flexibilidad, mediante el uso de punteros a void y el \textit{typecast} de los mismos en ejecución, aunque en el caso del \textit{parser}, solo se usó el tipo char*.

La implementación de la misma se hace mediante estructuras típicas de C enlazadas mediante punteros unidireccionales y el acceso a los datos de la pila se realiza mediante los conocidos métodos para operar con este tipo de datos (new, is empty, push, pop, etc.) 

Además de la adición de la estructura de pila para verificar esto se incluyeron más opciones de error en varias de las reglas gramaticales de modo que resultase más fácil la depuración del \textit{parser}. Esto reveló que había fallos en ejecución, ya que los resultados no eran los esperados. Este problema se debía a que la recursión se realizaba inicialmente mediante la posibilidad de que block derivase en otro block, así que se descompuso esta regla en un nuevo elemento body que añadía un paso más y abstraía los bloques de etiquetas y contenido pero a un nivel de granularidad algo más pequeño que body en sí. Tras la inclusión de este paso intermedio, el \textit{parser} aún no terminaba de comportarse como era esperado y hubo que incluir reglas específicas para comprobar tags con formatos especiales tales como $<$?xml version$>$ o $<$!DOCTYPE$>$. Por último se comprobó que el \textit{parser} no contemplaba inicialmente tags autocerrados en su forma $<$tag$>$$<$/tag$>$ es decir, sin contenido. Al corregir estos detalles, cualquier pieza musical era transformada a hechos musicales a la perfección.

\section{Iteración 2}

El prototipo objetivo de esta iteración es una primera versión del módulo de procesado en ASP. Para ello se fijan una serie de restricciones que simplifican algunos problemas detectados durante el análisis. Inicialmente no se tendrán en cuenta las notas adyacentes en la partitura ni subdivisiones de las mismas, no obstante y de cara a futuras iteraciones, se establece una figura base como período de análisis horizontal de la melodía y una figura base de subdivisión, aunque para esta iteración ambas se establecerán a negra (\textit{quarter note}) de modo que se asignará el acorde correspondiente a cada golpe rítmico de la partitura teniendo en cuenta las múltiples voces que la componen de manera vertical.

Para empezar se ha modificado el parser y se ha incluido una opción para subdividir la partitura de forma automática en base a la nota más breve de la partitura o forzar toda la partitura a un solo tipo de figura omitiendo aquellas figuras de menor duración, aunque esto lleve a resultados incorrectos musicalmente. La salida deseada y el formato de entrada para el módulo de procesado ASP usan el formato "note(voz, tono, tiempo)" siendo voz el número de la voz que interpreta la nota, tono un valor numérico asociado a la nota, no en frecuencia sino calculado como el número semitonos a partir de una nota base A0, es decir, el La más grave que puede interpretar MIDI; y tiempo la posición de la nota en la partitura.

Para el desarrollo del módulo de procesado ASP se sigue el proceso conocido como \textit{Generate and Test}, muy habitual en el paradigma. Como su nombre indica, consta de dos partes: generación y prueba. En la generación se usan reglas que definen todas los resultados posibles del problema para despues, en la fase de prueba, restringirlos mediante reglas que prohíben total o parcialmente ciertas combinaciones al exponerse a hechos lógicos aportados por el \textit{parser}. Esta metodología se usará desde el principio en cada iteración para desarrollar la parte en ASP correspondiente ya que aunque las reglas de generación permanecerán intactas, estas serán revisadas por si fuera necesario cambiarlas o añadir alguna regla nueva.

En esta primera aproximación, las reglas de generación establecen los posibles acordes que compondrán la solución y se realiza una asignación acorde-unidad rítmica (como ya se comentó, la unidad rítmica base inicialmente será la negra) en base a las notas presentes para un instante dado en cada voz. Para realizar esto es necesario interpretar cual es el grado de cada nota presente en la escala de la pieza, (de nuevo, por restricciones iniciales se considera Do mayor) ya que de este modo podremos especificar una serie de reglas que abstraigan el hecho de qué nota suena en si y lo sustituyan por qué grados aparecen en cada momento dado de la partitura, pudiendo deducir así qué acorde es el más correcto de entre todas las posibilidades, a más notas presentes de la escala correcta, menos posibilidades, de aparecer notas no pertenecientes a la escala, no se podrá resolver el acorde y el programa se detendrá con un error.

La fórmula empleada para derivar los grados de la escala es $[(valor-base) mod 12]$ siendo valor el valor númerico asignado a la nota y base la nota de la que se parte para calcular los grados. Esta expresión resulta en los semitonos de distancia entre la nota base y la nota actual. La operación de módulo es necesaria para abstraer la octava a la que pertenece la nota, aunque más adelante esto pueda ser un hecho relevante. Con una comparación directa del valor obtenido con una distribución dada según el modo (Mayor o Menor) de la escala es posible asignar los grados correspondientes a cada nota. Por último se generan hechos que indican cuantas voces toman parte en la pieza, así como los tiempos en los que suena al menos una nota.

Se crearon dos ficheros a mayores que especifican los acordes, otorgándole un nombre en notación numérica romana y creando múltiples predicados que especifican qué grados de la escala pertenecen a dicho acorde. De este modo no es necesario especificar en ningún momento cuantas notas tendrán los acordes con los que se trabaje, y aunque principalmente se trabajará con acordes de tres notas, esto permitirá ampliar la complejidad en el futuro al incluir acordes de dos o cuatro notas. Teniendo en cuenta los posibles acordes, y mediante una restricción de cardinalidad, se generan todas las soluciones iniciales, haciendo que solo se asigne un acorde a cada tiempo en el que suena al menos una nota. A mayores se especifica una restricción de integridad que anula cualquier solución en la que, para un tiempo en el que esté presente una nota concreta, el acorde asignado en la solución comprobada no contenga esa nota. 

A mayores se ha incluído como parte de esta iteración, el diseño e implementación de un pipeline en python que automatice el proceso de \textit{parseo} y de procesado mediante el módulo ASP. Dicho pipeline incluye la posibilidad de alternar entre los modos mayor y menor mediante una opción. En este primer prototipo no se han utilizado \textit{wrappers} para trabajar con \textit{clingo} directamente a través del pipeline, por ello se implementó una pequeña funcionalidad de interpretación de la salida del módulo ASP a un vector de soluciones. En futuras iteraciones, se añadirá más funcionalidad a dicho pipeline, como más opciones de entrada o diferentes representaciones de las soluciones ofrecidas por el módulo ASP.

Se han realizado pruebas individuales a los módulos \textit{parser} y ASP, además se han realizado pruebas al pipeline creado, sirviendo de este modo de pruebas de integración de los diferentes módulos y de la herramienta al completo.

Para el \textit{parser} se han probado múltiples ficheros de entrada así como las distintas opciones que contempla. No presenta problemas para ficheros MusicXML estándar generados por Musescore y programas similares, ante la ausencia de fichero, ficheros incompletos, ficheros XML que no sean MusicXML o ficheros que directamente no sean XML el \textit{parser} detiene la ejecución y produce el mensaje de error adecuado en cada caso. En cuanto a las tres opciones que presenta esta primera versión, la opción de ayuda -h detiene la ejecución independientemente de la presencia de otros parámetros y muestra el modo de uso del módulo parser por sí solo, la opción para especificar el fichero de salida -o funciona correctamente en cualquier caso, excepto cuando el directorio especificado para almacenar el fichero de salida no existe. En la siguiente iteración se planteará la posibilidad de dejar esto de este modo intencionadamente o si forzar la creación de los directorios necesarios. Por último la opción de subdivisión -s, que para este primer prototipo ha de ser especificada manualmente o dejada en su valor por defecto funciona para cualquier valor. Esto no es del todo correcto ya que la subdivisión solo debería ser posible para potencias de 2, en el segundo prototipo se corregirá esto.

El módulo ASP no presenta una gran complejidad de prueba ya que solo depende del fichero de entrada. Si este no está presente no produce resultado alguno, y en caso de estarlo y no ser correcto, produce un resultado de insatisfacibilidad. En caso de existir alguna solución esta es correctamente devuelta por pantalla.

Por último, en las realizadas para el pipeline se analizaron las diferentes opciones y casos de entrada de la herramienta. Ante los casos erróneos donde el módulo \textit{parser} falla y no genera fichero LP de salida, el pipeline no detiene la ejecución y llama al módulo ASP igualmente, aunque este falla al no tener un fichero de hechos lógicos con el que funcionar, no obstante el pipeline si que debería comprobar esto y por tanto se corrigió. Con ficheros de entrada adecuados el pipeline realiza una ejecución completa y recoge los resultados ofrecidos por el módulo ASP, ofreciendo por pantalla una representación más amigable así como un resultado de satisfacibilidad. Las opciones de esta primera versión del pipeline solo fueron dos, una que permite cambiar el modo de la amronía entre mayor y menor (-m major|minor) y otra que permite cambiar el número máximo de soluciones ofrecidos por el módulo ASP (-n N). La primera funciona como es debido, restringiendo los dos posibles valores de la opción a los especificados y deteniendo la ejecución en caso de encontrar un valor incorrecto, mientras que la segunda restringe también correctamente los valores de N a únicamente enteros y no hace falta mayor comprobación ya que cualquier valor entero es válido como opción para el módulo ASP.

Este primer prototipo se completó en dos semanas, siendo los mayores problemas de la iteración la falta de soltura con el paradigma y el lenguaje, así como detalles técnicos de falta de librerías y software. 


\section{Iteración 3}

El objetivo de esta iteración es completar el desarrollo de un segundo prototipo de la herramienta que incluya las siguientes mejoras: Subdivisión real y automatizada de las notas de la pieza a la longitud de la nota mínima presente en la partitura, especificación en el pipeline de la longitud del tiempo de análisis horizontal, inclusión de la posibilidad de realizar dicho análisis horizontal en ASP y flexibilización de los resultados del módulo de armonización, es decir, en vez de prohibir las soluciones erróneas, se anotarán los errores en la partitura y se escogerá aquella solución que minimice el número de errores. Además se incluirán en este prototipo las correcciones a los errores de ejecución de los módulos no corregidos en el primero.

Se modificó el \textit{parser} substancialmente ya que este imprimía a un fichero según procesaba las notas. Esto no planteaba problema alguno si la subdivisión se especificaba de antemano mediante el parámetro correspondiente, pero sí que resultaba complicado mantener esta aproximación si la unidad de subdivisión debía calcularse al mismo tiempo que se procesaba la partitura en MXML. Se plantearon dos soluciones, o bien incluir en el pipeline en python un análisis previo a la conversión de MXML a hechos en ASP que dedujese cual era la nota de menor longitud y la usase como parámetro en la llamada al \textit{parser} o bien se modificaba el comportamiento del anterior para realizar simultáneamente ambas tareas. 

Se optó por la segunda opción por motivos de coherencia con el sistema, es decir, no incluir funcionalidad innecesaria y replicada en el pipeline, cuya tarea es simplemente manejar las entradas y salidas de los diferentes módulos, y por motivos de eficiencia, ya que como se ha mencionado no hay necesidad de procesar el mismo fichero dos veces, siendo una de ellas en un lenguaje interpretado en vez de compilado, lo que añadiría un sobrecoste temporal evitable.

Los cambios implementados en el \textit{parser} implicaron principalmente incluir un nuevo tipo de dato nota para almacenar la información de las notas de la partitura y una nueva pila que contuviese las notas extraídas del MXML. Una vez procesado todo el fichero de entrada, hallada la nota más breve y almacenadas las notas en la pila, ésta se vacía y se imprimen en el fichero de hechos lógicos de salida teniendo en cuenta la subdivisión pertinente, bien sea esta la calculada o la especificada por parámetro. En caso de especificar una subdivisión no válida, es decir, de mayor longitud que alguna de las notas presentes en la partitura, se imprime por pantalla un mensaje de error y la nota no es subdividida. Esto produce comportamientos no deseados a la hora de realizar la deducción de la armonía, ya que es necesario trabajar siempre con piezas con notas de longitudes iguales a lo largo de toda la partitura.

El módulo de armonización incluye una nueva constante que indica la longitud del intervalo de tiempo mínimo de análisis armónico horizontal, a su vez es posible especificar en la llamada al módulo el valor de esta constante. Se ha modificado, por tanto, la regla que restringe las posibles soluciones para analizar en dichos intervalos de tiempo. Además dicha regla se ha suavizado y en vez de ser una restricción de integridad, esta activa un nuevo predicado error(voz, grado, tiempo) que indica los grados erróneos presentes en la partitura que no encajan con el acorde asignado para la solución. Posteriormente se realiza un proceso de optimización consistente en la minimización de el número de estos predicados de error, es decir, aquellas soluciones con menor número de errores serán las óptimas. Además en caso de no encontrar una solución con cero errores, los errores aparecen también en la salida para que el usuario pueda conocer qué tiempos contienen notas equivocadas.

Se han incluido en el pipeline opciones tanto para indicar al \textit{parser} una subdivisión específica como para indicar al módulo de armonización el intervalo horizontal de armonización. Se han implementado en el pipeline, con vistas al futuro de módulo de salida, una serie de clases para almacenar los resultados y poder devolverlos más tarde en el formato más conveniente. Error y Chord son clases para almacenar y representar los acordes asignados en la solución así como los errores presentes en la partitura, es decir, aquellas notas que no encajan con los acordes asignados. A su vez, HaspSolution es una clase orientada a almacenar y representar soluciones completas incluyendo una colección de objetos Chord y otra de objetos Error, así como el grado de optimizazión de dicha solución. Por último, la clase ClaspResult almacena la salida sin procesar de una ejecución del módulo ASP, aunque cuando esta es instanciada, dicha salida se procesa y se crean una colección de objetos HaspSolution conteniendo sólo aquellas soluciones que alcancen un cierto valor de optimización. La relación entre las clases así como sus métodos se detallan en \textbf{el diagrama de clase pertinente}.

Se han continuado con las pruebas a los diferentes módulos, dando por válidos los resultados de las pruebas de la anterior iteración se procedió a probar las nuevas funcionalidades de cada módulo.

En el \textit{parser} se comprobó que la autosubdivisión funcionaba correctamente al introducir partituras con diferentes longitudes de notas, además en caso de forzar una subdivisión incorrecta se producía el mensaje de aviso adecuado explicando el problema. Si la subdivisión forzada era correcta funcionaba como debía. El nuevo tipo de datos, nota, así como la pila para almacenar las diferentes notas de la partitura fueron probados junto con esta nueva funcionalidad, ya que fueron implementados expresamente para su funcionamiento.

En el módulo ASP se probaron piezas más y menos complejas, algunas creadas de forma artificial para producir errores y se vio que el nuevo análisis funcionaba y detectaba dichos errores, no obstante seguía produciendo soluciones válidas marcando aquellas notas erróneas.

En cuanto al pipeline se comprobó que las nuevas opciones en la llamada a éste funcionasen correctamente, ya que las dos nuevas opciones son simplemente parámetros que se pasan más adelante al módulo \textit{parser} o al módulo ASP, bastó con realizar una comprobación de validez del valor pasado en la llamada. Se probaron además las cuatro clases de almacenamiento de las soluciones diseñadas e implementadas para este prototipo, siendo ClaspOutput la única que tuvo que ser probada más exhaustivamente ya que es la que realiza el procesado de la salida del módulo ASP, el resto simplemente cuentan con constructores y funciones de representación textual. 

\section{Iteración 4}

El tercer prototipo del proyecto, correspondiente a esta iteración pretende no solo refinar la armonización como el anterior sino ampliar funcionalidad de la herramienta. Se ha incorporado un módulo de salida escrito en python al cual el pipeline se encarga de pasarle los datos formateados correctamente para que dicho módulo, haciendo uso del toolkit \textbf{music21}, exporte al formato deseado. Se ha optado por este módulo principalmente por la cantidad de opciones de salida que posee, y aunque lo ideal será exportar un fichero MusicXML, la idea de poder generar PDF, MIDI o Lilypond resulta más que atractiva.

El módulo ASP se ha aumentado para incluir generación de notas en un nñumero de voces adicionales que puede ser especificado por parámetro. Además se creó un fichero de conversiones encargado de traducir valores de notas a grados, octavas y viceversa. De este modo los grados generados durante el procesado de la partitura pueden ser traducidos de vuelta a un valor de nota para que el módulo de salida llamado desde el pipeline reconstruya la pieza. Para la generación de notas en las nuevas voces se ha impuesto una única restricción fuerte, que dos notas consecutivas no realicen un salto melódico de dos octavas o más, mientras que mediante predicados de minimización se controla la cantidad de saltos de una quinta realizados por una misma voz. Junto con estas adiciones, se han incluído dos nuevos acordes a la lista de posibles acordes deducidos por el módulo, siendo estos la subdominante séptima (IV7) y la dominante séptima (V7) tanto de los modos mayor como menor.

Se ha diseñado implementado una nueva clase de almacenamiento Note y se ha incluido un método a HaspSolution que transforma una solución al tipo de dato requerido por el \textit{toolkit} music21 para producir los diferentes formatos de salida.

El módulo de salida toma un objeto ClaspResult como entrada y haciendo uso del método de una de los objetos HaspSolution contenidos en él, representa dicha solución en el formato adecuado.

En el \textit{pipeline} se ha incluido una opción para especificar el número de voces adicionales que deben ser añadidas y otra opción para especificar el formato de salida. Este componente ahora se encarga también de llamar al módulo de salida.

\section{Iteración 5}

Por necesidades del módulo de salida, el \textit{parser} cuenta con nueva funcionalidad y ahora almacena metadatos de la partitura, como el tempo, el título o el autor de la misma en un fichero de texto que después será utilizado por el módulo de salida para reconstruir la pieza lo mejor posible. Si bien no es elegante, la otra opción sería traducir estos datos a hechos lógicos para poder usarse más adelante, lo cual no tiene mucho sentido y por tanto se ha optado por el fichero.