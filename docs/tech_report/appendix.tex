\newpage

\chapter{Installation}
\label{chap:installation}
In it's current state, it's advised to install the tool in a Linux environment. This is mostly because of the use of Flex and Bison libraries, that are easily compiled and used in that kind of environments.

The requirements are Python 2.7 and the latest versions of Flex and Bison (as well as a C compiler). Other than that, gringo 3.0.5 and clingo 3.0.5 are needed, those can be downloaded at the Potassco Group's Sourceforge page \cite{potasscoweb} as well as the music21 module in it's latest version. All these tools and libraries should be accessible on the System's \texttt{PATH}

It is advised to install any tools required to compile or visualize the desired output formats before installing music21 so it can be properly attached to them without any further configuration.

Of course a score edition tool is also advised, it is completely optional but it's convenient. Musescore2 was used for the tests and demos as it's Open Source and free, but any other similar editor that can handle MusicXML files will work.

The tool can finally be executed by calling it's main entry point on a command line, and any of the parameters as well as it's usage explanation can be checked with the \texttt{-h} or \texttt{--help} options.