\chapter{Background}
\label{chap:background}
\vspace{0.5cm}

%%%%%%%%%%%%%%%%%%%%%%%%%%%%%%%%%%%%%%%%%%%%%%%%%%%%%%%%%%%%%%%%%%%%%%%%%%%%%%%%
% Objetivo: Contar cómo estaba la situación antes de empezar,                  %
%           todo lo que se hizo para familiarizarse con las tecnologías,       %
%           casarlas, etc.                                                     %
%%%%%%%%%%%%%%%%%%%%%%%%%%%%%%%%%%%%%%%%%%%%%%%%%%%%%%%%%%%%%%%%%%%%%%%%%%%%%%%%

\section{State of the Art}
\label{sec:state-of-the-art}

\lettrine{M}ost of the research done in computer music is extracted from the relationship between musical theory and mathematics. This relationship is fairly easy to understand and model in a mathematic way.

There are two main lines of research in computer music but hey often overlap since the research matter is the same, despite changing the point of view. These lines are Composition Assistance Tools and Composing-Oriented Intelligent Systems. Haspie lies in a point between these two lines of research: It's a tool to assist the composition but it's able to understand and develop any score given to it.

\subsection{Answer Set Programming}
\label{subsec:asp}

The core of the project is \textit{Answer Set Programming}\cite{Brewka:2011:ASP:2043174.2043195} (ASP from now on). The main module of the project has been developed through the use of the Stable Models of Gelfond \& Lifschitz\cite{Gelfond88thestable} and non-monotonic logic. ASP is a declarative programming language oriented to solve hard search problems, mostly NP-complex ones.

Since it's a declarative language, ASP programs only requires the inference rules and constraints of the problem to solve any instance of it, these instances need to be previously translated to ASP facts. ASP uses a grounder to expand these facts using the inference rules of the program to calculate the solutions that will be later pruned through the previously mentioned constraints, thus finding the solution (or multiple solutions if there were more than a single one). This methodology is known as \textit{generate and test}.

The ASP Tools developed by the Potassco Group\footnote{http://potassco.sourceforge.net/} include, among others, a \textit{grounder} (Gringo) and a \textit{solver} (Clasp), but these tools are packed into a single program called Clingo, that does both grounding and solving.

\subsection{MusicXML}
\label{subsec:formats}
The format that haspie uses as input for the musical scores is MusicXML. MusicXML, MXML or \textit{Music Extensible Markup Language} is an extension of the XML format used to represent occidental music. No only it does include the score information but also includes how it should be represented on paper (margins, font sizes, musical notes position coordinates in the sheet, etc.)

It uses xustom XML tags to group different levels of musical information together such as the general piece data, parts, measures and such. (See Figure \ref{fig:nota_musicxml}). It's a very rich format but due to it's complexity and very easy parsing properties it's meant for computer use more than for human writing and reading.

\begin{figure}[h!]
	\centering
	\begin{Verbatim}[frame=single]
<note default-x="74.65" default-y="-25.00">
	<pitch>
		<step>A</step>
		<octave>4</octave>
	</pitch>
	<duration>1</duration>
	<voice>1</voice>
	<type>quarter</type>
	<stem>up</stem>
</note>
	\end{Verbatim}
	\caption{Sample A note represented in MusicXML}
	\label{fig:nota_musicxml}
\end{figure}


\subsection{Software}
\label{subsec:software}
\subsubsection{Tools and Libraries}
\begin{itemize}
	\item \textbf{Flex and Bison} are Unix utilities that allow the creation of fast text parsing tools.
	\item \textbf{Music21} is a suite of tools that helps students and musicians alike to query info about well-known musical pieces. Not only it includes a very complete musical database but it also includes libraries to help in programmatic music composition.
\end{itemize}

\subsubsection{Musescore2}
Musescore2 is the editor of choice to work along haspie. It's open source and free and allows to import and export all of the common music representation formats, even synthesizing the score sound in MIDI and other sound formats. 

\subsubsection{Intelligent Systems}
In the AI field there has been many approaches to the musical composition subject form many different points of view of this field.
\begin{itemize}
	\item \textbf{EMI and Emily Howell} Developed by David Cope\cite{experiments-musical-intelligence}, EMI or Experiments in Music Composition, is a system capable of identifying the style of a musical piece and complete any required section of it following the same style. Cope's work studies the possibility of using grammars along with dictionaries in musical composition. After the work on EMI, it derived into the software known ad Emily Howell. Emily Howell uses EMI to create and update it's musical database but has it's own interface through which the user can give feedback and modify the current composition. Cope polished Emily Howell with his own musical style to create and compose many music albums that where later published.
	\item \textbf{ANTON}\cite{anton-composing} is a rhytmic, melodic and armonic composition system based in Answer Set Programing. ANTON is capable of composing small musical pieces from scratch or parting from a given score. It uses the renaissance style of Giovanni Pierluigi da Palestrina since it's a well defined set of musical composition rules that can be translated to ASP rules.
	\item \textbf{Vox Populi}\cite{vox-populi} uses genetic algorithms to compose music in real time. This system starts from a population of chords codified throught the MIDI protocol to then perform the required genetic steps (crossings, fittings, mutations, etc) selecting the best of them each iteration through pure physical methods regarding sound frequencies.
	\item \textbf{CHORAL} is an expert system that works as a harmonizer with the classic style of Johann Sebastian Bach. The rules used by the system represent the musical knowledge from the different points of view of the choral group. To achieve the harmonization, the program uses a generate and test with backtracking algorithm.
	\item \textbf{CHASP} is a tiny tool created by the Potassco Group to calculate chord progressions through Answer Set Programming starting from scratch, allowing the user to specify key and length of the piece. Despite not having an input file, it allows exporting the result and post-processing it to imprint different rhythmic styles to the composed piece.
\end{itemize} 




