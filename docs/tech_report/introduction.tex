\chapter{Introduction}
\label{chap:introduction}

%%%%%%%%%%%%%%%%%%%%%%%%%%%%%%%%%%%%%%%%%%%%%%%%%%%%%%%%%%%%%%%%%%%%%%%%%%%%%%%%
% Objetivo: Exponer de qué va este proyecto, sus líneas maestras, objetivos,   %
%           etc.                                                               %
%%%%%%%%%%%%%%%%%%%%%%%%%%%%%%%%%%%%%%%%%%%%%%%%%%%%%%%%%%%%%%%%%%%%%%%%%%%%%%%%

\lettrine{M}usic Theory learning has been stuck in the same old-fashioned methods and systems for years. These methods are based on exercises and repetition, meant to train the ear and become fluent in composition or in solving the proposed exercises. Harmony learning is one of the most important steps in music, since being able to analyse scores in this way is vital for its comprehension, later interpretation and further development. Haspie aims to help Harmony students achieve a better understanding of the matter and lets them experiment earlier with composition from an harmonic point of view.

Constraint Logic fits well with the problem since the harmony rule set used in the most basic levels taight in music schools hasn't changed since the origins of the subject, it's a definite set of rules, not very big and quite strict. Simply translating this rule set to ASP constraints and being able to extract the fatcs and knowledge from any score, the tool is able to detect any mistake and propose solutions, as well as filling in any blank sections of the score to create, in the end the harmony of the piece.

The great advantage that Answer Set Programming provides is that there is no need to create nor optimise a solution searching algorithm, since it only needs these harmony rules. Answer Set Programming not only offers this simplicity and power, but also flexibility since a change in the rules or in the optimization configuration can lead to very different results and adapt better to the composition style of the user.

 
