%%%%%%%%%%%%%%%%%%%%%%%%%%%%%%%%%%%%%%%%%%%%%%%%%%%%%%%%%%%%%%%%%%%%%%%%%%%%%%%%

\begin{abstract}
\thispagestyle{empty}
Existen múltiples aproximaciones a la música dentro de la ingeniería de computadores, desde sistemas de ayuda y guía a la composición integrados en herramientas de escritura de partituras a conjuntos software que crean piezas musicales en estilos concretos partiendo de cero o que componen piezas sencillas, las ofrecen a un usuario para su evaluación y las mutan iterativamente en una suerte de evolución para lograr variaciones que satisfagan cada vez más al usuario.
Lo que se plantea aquí es una herramienta de ayuda inteligente capaz de comprender la armonía presente en una partitura y completar tramos de la misma, llegando incluso a poder crear nuevas voces de la nada que sean correctas desde un punto de vista armónico. Esto puede servir como una herramienta individual para ayudar al estudiante de armonía a comprender mejor la materia o al compositor novel a explorar nuevas soluciones a la hora de incluír secciones en su pieza, aunque también puede servir como utilidad intermedia para ser combinada con otras herramientas de manipulación musical.
Pese a que el fin del proyecto es crear la herramienta en sí, sirve también como demostración empírica de la aplicabilidad de las técnicas de \textit{Answer Set Programming} al campo de la música. La gran ventaja que plantean estas técnicas es la independencia del sistema de definir cualquier algoritmo de búsqueda o heurística en la que basarse para hallar las posibles soluciones. La gran potencia de ASP consiste en poder definir, en muy pocas reglas, todo el conjunto de soluciones posibles del problema y restringirlas con otro pequeño conjunto, de este modo, cualquier cambio que se quiera realizar en los resultados puede ser fácilmente ajustado sin entender el funcionamiento del programa en sí.
\end{abstract}

%%%%%%%%%%%%%%%%%%%%%%%%%%%%%%%%%%%%%%%%%%%%%%%%%%%%%%%%%%%%%%%%%%%%%%%%%%%%%%%%
