\chapter{Evaluación}
\label{chap:evaluation}
\vspace{0.5cm}

%%%%%%%%%%%%%%%%%%%%%%%%%%%%%%%%%%%%%%%%%%%%%%%%%%%%%%%%%%%%%%%%%%%%%%%%%%%%%%%%
% Objetivo: Exponer los resultados objetivos del sistema                       %
%%%%%%%%%%%%%%%%%%%%%%%%%%%%%%%%%%%%%%%%%%%%%%%%%%%%%%%%%%%%%%%%%%%%%%%%%%%%%%%%

 \lettrine{E}{n} este capítulo se van a exponer los resultados de la evaluación del sistema por diferentes usuarios relacionados con las dos materias relativas al proyecto Música e Informática. Cada uno de los expertos aportará una pieza de su elección y propondrá una modificación a la misma para ver como se comporta el sistema cuando tenga que completar la partitura escogida. Los expertos que formaron parte de la evaluación fueron:
 
 \begin{center}
 	\begin{tabular}{ | l | c | c | c |}
 		\hline
 		Nombre & C. Musicales & C.Informáticos & Pieza Escogida \\ \hline \hline
 		Experto A & Medios & Altos &  Death by Glamour \\ \hline
 		Experto B & Altos & Bajos & Menuet \\ \hline
 		Experto C & Altos & Altos & Greensleves \\ \hline
 		Experto D & Bajos & Medios & Joy to the World \\ \hline
 	\end{tabular}
 \end{center}
 
 Además se incluye una breve explicación de cada una de las piezas, los motivos por los que fueron propuestas así como la modificación a realizar en cada una de ellas.
 
 \begin{itemize}
 	\item \textbf{Death by Glamour:} Compuesta por Toby Fox para la banda sonora del videojuego ``Undertale'', Death by Glamour es una pieza interesante por su frenética melodía y por estar pensada para ser interpretada en un piano a cuatro manos. Se sugiere vaciar algunos tramos de diferentes voces.
 	\item \textbf{Menuet:} Famosa pieza de Johann S. Bach, destaca por su simpleza y es interesante para ver como funciona el programa ante compases ternarios. Se sugiere añadir una voz nueva de tesitura más aguda a la presente en la pieza.
 	\item \textbf{Greensleves:} Supuestamente compuesta por Enrique VIII, esta archiconocida partitura presenta una polifonía coral a cuatro voces, ideal para comprobar las capacidades de armonización del sistema. Se sugiere eliminar secciones grandes de la voz solista y ver cómo la completa.
 	\item \textbf{Joy to the World:} Conocido villancico, sería interesante escuchar una reinterpretación de la voz más grave para la pieza, ya sea completando secciones o bien añadiendo una voz de bajo.
 \end{itemize}

